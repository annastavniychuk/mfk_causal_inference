

\section{Инструментальные переменные в экспериментах}


\begin{frame}{Пример с эффектом от таблетки}
\begin{itemize}
    \item Мы тестируем новое лекарство против covid и раздали в пневмонийном отделении половине настоящую таблетку другой половине плацебо
    \item мы наблюдаем процент, что среди тех, кому дали таблику процент выздоровления повышается на 7 п.п.\footnote{Цифры приведены в демонстрационных целях}
    \item 70 процентов пациентов таблетку приняли
    \item Выздоровление тех людей, кто принял лекарство ниже, чем тех, кто не принял лекарство! Как так получилось?
    \item Какова эффективность лекарства? \visible<2->{10 п.п.}
\end{itemize}
\end{frame}

\begin{frame}{Оценка эффекта размера семьи}
\begin{itemize}
    \item Хотим оценить эффект второго ребенка на благосостояние (без панельных данных)?
    \item Мы знаем, что доход людей у которых родились близнецы в среднем на 4 процентых пункта ниже.\footnote{Цифры приведены в демонстрационных целях}
    \item Также мы знаем, что люди добровольно заводят второго ребенка с вероятностью 75%
    \item Какой средний эффект второго ребенка на доход? \visible<2->{16 п.п.}
\end{itemize}
\end{frame}

\begin{frame}{Оценка эффекта службы в армии на будущие доходы}
\begin{itemize}
    \item Каждого 15-ого мужчину случайным образом призывают на службу во Вьетнаме
    \item Доход призванных людей у которых родились близнецы в среднем на 4 процентых пункта ниже.\footnote{Цифры приведены в демонстрационных целях}
    \item Бывают люди, которых невозможно призвать по состоянию здоровья, бывают люди, которые служат в армии добровольно. Мы наблюдаем только исход лотереи и факт службы в армии.
    \item Как оценить эффект службы в армии на будущие доходы?
\end{itemize}
\end{frame}

\begin{frame}{Терминология}
\begin{itemize}
    \item Инструмент Z (предложение таблетки, факт близнецов)
    \item Переменная интереса T (принятие лекарства, количество детей)
    \item Исход Y (температура, доход)
\end{itemize}
\end{frame}


\begin{frame}{Эффекты}

Как устроен мир: Z -> T -> Y

\begin{itemize}[<+->]
    \item Reduced form (Intention to treat): Z -> Y
    \item First stage: Z -> T
    \item Second stage: T -> Y
\end{itemize}

\end{frame}



\begin{frame}{Структурный подход против reduced form}

Плюсы структурного подхода:
\begin{itemize}
    \item Можно получить информацию, которая другим способом недоступна (эффективность лекарства)
    \item Можем оценить параметры теоретических моделей
    \item Можно фальсифицировать теорию (e.g. закон спроса)
    \item Можно ответить на вопрос <<почему>> -- изучить передаточный механизм
\end{itemize}

Минусы структурного подхода:
\begin{itemize}
    \item Получаем результаты того, на что можем непосредственно повлиять
    \item Более слабые предпосылки не требующие корректности теории
    \item Слабые -- значит надежные. Работают даже если теория не верна
    \item Более устойчивые результаты: мало свободы у исследователя повлиять на результат функциональной формой
\end{itemize}
    
\end{frame}


\begin{frame}{Обозначения} 
\begin{itemize}
    \item $Y_{00}, Y_{10}, Y_{01}, Y_{11}$ -- Зависимая переменная (\textbf{potential outcomes})
    \item $T_1$, $T_0$ -- Переменная интереса (теперь тоже с потенциальными исходами)
    \item $Z$ -- Инструментальная переменная (\textbf{instrumental variable})
    \item $X$ -- Независимые переменные (\textbf{Covariates})
    \item Мы наблюдаем только $(Y, P, X)$, где $T = ZT_1 + (1-T)T_0$ -- \textbf{observed treatment} $Y = \texttt{длинная формула}$ -- \textbf{observed outcomes}
\end{itemize}
\end{frame}

\begin{frame}{Предпосылки}

\begin{itemize}
    \item $(Y_{ij}, T_i, X) \perp Z$ -- рандомизация (exclusion restriction)
    \item SUTVA
    \item<2-> $P(T_1 \geq T_0) = 1$ -- монотонность %hide at all
    \begin{itemize}
        \item Мы хотим, чтобы бывало такое, чтобы никогда не было такого, что  $T_1 < T_0$, но бывало такое, что $T_1 > T_0$
        \item Бывает одностороннее неповиновение тритементу (one-sided noncompliance)
        \item Бывает двустороннее неповиновение тритементу (two-sided noncompliance)
    \end{itemize}

Когда выполнены все эти предпосылки, мы говорим, что эффект \textbf{идентифицирован}.
\end{itemize}
\end{frame}


\begin{frame}{Подробнее про предпосылки}
\begin{itemize}
    \item $P(T_1 \geq T_0) = 1$ -- монотонность. На самом деле возможных случает в только 4
    \begin{itemize}
        \item Always takers: $T_1 = 1$, $T_0 = 1$
        \item Compliers: $T_1 = 1$, $T_0 = 0$
        \item Never takers: $T_1 = 0$, $T_0 = 0$
        \item Defiers: $T_1 = 0$, $T_0 = 1$
    \end{itemize}
\end{itemize}
\pause
\begin{itemize}
        \item Two-sided noncompliance: no Defiers
        \item Two-sided noncompliance: no Defiers and no Always takers
    \end{itemize}
\end{frame}

% \begin{frame}{Подробнее про предпосылки}
% \begin{itemize}
%     \item $P(T_1 \geq T_0) = 1$ -- монотонность. На самом деле возможных случает в только 4
%     \begin{itemize}
%         \item Always takers: $T_1 = 1$, $T_0 = 1$
%         \item Compliers: $T_1 = 1$, $T_0 = 0$
%         \item Never takers: $T_1 = 0$, $T_0 = 0$
%         \item Defiers: $T_1 = 0$, $T_0 = 1$
%     \end{itemize}
% \end{itemize}
% \pause
% \begin{itemize}
%         \item One-sided noncompliance: Нам 
%         \item Compliers: $T_1 = 1$, $T_0 = 0$
%         \item Never takers: $T_1 = 0$, $T_0 = 0$
%         \item Defiers: $T_1 = 0$, $T_0 = 1$
%     \end{itemize}
% \end{frame}

% формулы!

\begin{frame}{Пример одностороннего неповиновения: encouragement design}

\begin{enumerate}
    \item Пример с таблеткой
    \item Пример с детьми
\end{enumerate}

\end{frame}

\begin{frame}{Пример двустороннего неповиновения}

\begin{enumerate}
    \item Пример с армией
\end{enumerate}

\end{frame}

\begin{frame}{Формула}

\begin{gather}
    \text{ITT}_T = \bar T_{Z=1} - \bar T_{Z=0} = P(\text{compliers}) \\
    \text{ITT}_Y = \bar Y_{Z=1} - \bar Y_{Z=0} \\ % = разлоджить
    \text{TE} = \frac{\text{ITT}_Y}{\text{ITT}_T} = E(\tau|\test{Compliers}) = \text{LATE}
\end{gather}
\end{frame}

\begin{frame}{Откуда брать инструментальные переменные?}
\begin{itemize}
    \item Из опыта чужих исследований
    \item Миру известны десятки хороших инструментов. Каждый из них вызывает много дебатов
    \item Если вы придумали свой инструмент это тянет на хорошее оригинальное исследование.
\end{itemize}

\end{frame}


% \begin{frame}{ITT и Как это связано с обычной регрессией}

    
% \end{frame}

% примеры как теория помогает с этими предпосылками работать...



% \section{Гетерогенные эффекты}



% \begin{frame}{Двушаговая оценка}
% \begin{itemize}
%     \item Первый шаг
%     \begin{itemize}
%      \item На подвыборке $Z=1$ оценить модель $P_1(X)$
%      \item На подвыборке $Z=0$ оценить модель $P_0(X)$
%     \end{itemize}
%     \item Второй шаг
%     \begin{itemize}
%      \item Взять прогнозы $\hat P(Z, X)$ и использовать их в регрессии $\hat Y(\hat P, X)$
%     \end{itemize}
% \end{itemize}
% \end{frame}

% \begin{frame}{Предположения}
% \begin{itemize}
%     \item $(P_1, P_0, X) \perp Z$ -- рандомизация
%     \item $P(P_1 > P_0) = 1$ -- монотонность
%     \item Еще пара совершенно технических предпосылок
% \end{itemize}


% ЗАМЕТКИ:
%     Гетерогенные инстурменты на основе этой статьи https://sci-hub.se/10.1016/s0304-4076(02)00201-4
    
%     ОЧЕНЬ ХОРОШАЯ СТАТЬЯ, объясняет почему LATE оценка получается с инстурментами и как из нее ATE сделать https://sci-hub.se/https://www.cambridge.org/core/journals/political-analysis/article/beyond-late-estimation-of-the-average-treatment-effect-with-an-instrumental-variable/604E0803793175CF88329DB34DAA80B3
    
%     Вот тут статья обсуждает предопсылку монотонности в более общем случае, а не в бинарном. Но бинарный случай надо рассказать очень хорошо! https://faculty.smu.edu/millimet/classes/eco7377/papers/manski%20pepper%2000.pdf

% \end{frame}
