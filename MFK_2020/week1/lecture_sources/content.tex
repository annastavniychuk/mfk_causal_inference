\documentclass[handout,12pt]{beamer}
\usepackage{Lecture_Sources/packages/common_imports}
\usepackage{Lecture_Sources/packages/appearance}

\title{Причинно-следственные связи в данных}
\author[Георгий Калашнов, Ольга Сучкова]{Георгий Калашнов, Ольга Сучкова}
\date{\today}

\begin{document}

\begin{frame}
  \titlepage
  \begin{flushright}
     Вот почему попугаи за номером один, два и три были так похожи друг на друга: они были просто одним и тем же попугаем. (А. и Б. Стругацкие)
  \end{flushright}
\end{frame}


\begin{frame}{План на сегодня} 
\tableofcontents
\end{frame}

\import{sections/}{course_intro.tex}
\import{Lecture_Sources/}{experiments_intro.tex}


\begin{frame}{Что надо запомнить}
    \begin{itemize}
        \item Наша основная задача: оценить эффект от бинарного воздействия. Например от приема таблетки.
        \item Эффект воздействия может отличаться на разных подвыборках
        \item Чтобы получить несмещенную оценки нужно назначать таблетку случайно.
        \item Можно проверить, случайно таблетка назначалась, или нет
    \end{itemize}
\end{frame}

\begin{frame}{Про зачёт по курсу}
В качестве зачетной работы будет предложено разобрать эмприческое исследование и ответить на вопросы по нему. 
Пример - статья Vincent Pons «Will a Five-Minute Discussion Change Your Mind? A Countrywide Experiment on Voter Choice in France» American Economic
Review 2018, 108(6): 1322–1363.
\begin{itemize}
      \item Какой исследовательский вопрос интересует авторов? Что именно они хотят измерить?
    \item Какие данные используют исследователи для ответаа на этот вопрос? Какой метод используют и почему?
    \item Какой из полученных результатов отвечает на главный исследовательский вопрос? Проинтерпретируйте оценки.
    \item В уравнениях регрессии в качестве контрольных переменных используются резльтаты прошлых выборов. Является ли это «плохим контролем»?
\end{itemize} 
\end{frame}

\import{Lecture_Sources/}{citation_slides.tex}

\end{document}
