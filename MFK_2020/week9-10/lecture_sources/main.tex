% HOW TO PARAMETRIZE RELATIVE IMPORTS???

\documentclass[12pt]{beamer}
\setbeamercovered{dynamic}
\usepackage{Lecture_Sources/packages/common_imports}
\usepackage{Lecture_Sources/packages/appearance}


\title{Инструментальные переменные в модели Рубина}
\author[Георгий Калашнов, Ольга Сучкова]{Георгий Калашнов, Ольга Сучкова}
\date{\today}

\begin{document}

\begin{frame}
  \titlepage
  
\end{frame}




% нужно обобщить такие слайды в syllabus и просто bind и общий progress plan иметь
\begin{frame}{Обзор пройденного до сегодняшнего момента}
\begin{enumerate}
    \item Экспериментальные методы
    \begin{enumerate}
        \item Честный эксперимент
        \item Мера склонности (propensity score)
    \end{enumerate}
    \item Квазиэкспериментальные методы:
    \begin{enumerate}
        \item Панельные данные: разность разностей
        \item Разрывная регрессия
    \end{enumerate}
        \item Инструментальные переменные \textbf{(вы здесь)}
    \begin{enumerate}
        \item В экспериментах
        \item В панельных данных
        \item В разнывной регрессии
    \end{enumerate}
\end{enumerate}
\end{frame}

\begin{frame}{План на сегодня} 
\tableofcontents
\end{frame}


\import{Lecture_Sources/}{iv.tex}
% \import{Lecture_Sources/}{identification.tex}
\import{Lecture_Sources/}{bartik.tex}
\import{Lecture_Sources/}{fuzzy_regression_discontinuity.tex}
\import{Lecture_Sources/}{citation_slides.tex}



\end{document}
